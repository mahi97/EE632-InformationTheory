\documentclass[
  % all of the below options are optional and can be left out
  % course name (default: 2IL50 Data Structures)
  course = {{EE623 Information Theory}},
  % quartile (default: 3)
  quartile = {{Fall 2020}},
  % assignment number/name (default: 1)
  assignment = 10,
  % student name (default: Some One)
  name = {{Mohammad Mahdi Rahimi}},
  % student number, NOT S-number (default: 0123456)
  studentnumber = {{20208244}},
  % student email (default: s.one@student.tue.nl)
  email = {{mahi@kaist.ac.kr}},
  % first exercise number (default: 1)
  firstexercise = 1
]{aga-homework}

\usepackage{amssymb,latexsym,amsmath,amsthm}
\usepackage{amsfonts,rawfonts}
\usepackage{thmtools}
\usepackage{systeme}
\usepackage{mathtools,cancel}
\usepackage{tikz}
\usetikzlibrary{arrows,positioning} 
\usepackage{pgfplots} 
\usepackage[pdf]{graphviz}
\pgfplotsset{width=10cm,compat=1.9} 
 \usepgfplotslibrary{external}

\tikzexternalize 

\begin{document}

\exercise
\subexercise Compute the size $|T_{P_{XY}}|$ with exponential approximation.
\\\\
By the definition we have $|T_{P_{XY}}|$ equal to multinomial probability.
\begin{equation} \label{eq1}
\begin{split}
|T_{P_{XY}}| = \binom{n}{P_{x_1y_1}, P_{x_1y_2}, ... P_{x_1y_n}, P_{x_2y_1}, P_{x_2y_1}, ...P_{x_ny_{n-1}}, P_{x_ny_n}}\\
\textbf{By using Stirling's Approximation for large } n \Rightarrow |T_{P_{XY}}| \doteq e^{nH(P_{XY})}
\end{split}
\end{equation}

\subexercise Prove: For each pair $(x_n; y_n) \in T_{P_{XY}}$, it is true that $x_n \in T_{P_X}$
\begin{equation} \label{eq2}
\begin{split}
(x_n; y_n) \in T_{P_{XY}} & \Rightarrow \left| \hat{P}_{x^n, y^n}(a,b) - P_{XY}(a,b)\right| \le \delta\ \textbf{and $\delta \rightarrow 0$ as $n \rightarrow \infty$}\\
& \Rightarrow P_{XY}(a,b) - \delta \le \hat{P}_{x^n, y^n}(a,b) \le P_{XY}(a,b) + \delta\\
\textbf{Sum for all possible Y to get X marginal} & \Rightarrow \sum_Y (P_{XY}(a,b) - \delta) \le \sum_Y \hat{P}_{x^n, y^n}(a,b) \le \sum_Y (P_{XY}(a,b) + \delta)\\
& \Rightarrow P_{X}(a) - |Y|\delta \le \hat{P}_{x^n}(a) \le P_{X}(a) + |Y|\delta\\
& \Rightarrow \left| \hat{P}_{x^n}(a) - P_{X}(a)\right| \le |Y|\delta = \gamma \textbf{and $\gamma \rightarrow 0$ as $n \rightarrow \infty$}\\
& \Rightarrow x_n \in T_{P_{X}}
\end{split}
\end{equation}
\subexercise Compute the size $|T_W(x^n)|$ with exponential approximation.

We can find size of $|T_W(x^n)|$ by computing average size of $|T_{P_{XY}}|$ for every $T_{P_X}$.
\begin{equation} \label{eq3}
\begin{split}
|T_W(x^n)| = |T_{P_{Y|X}}| = \frac{|T_{P_{XY}}|}{|T_{P_{X}}|} \doteq \frac{e^{nH(X,Y)}}{e^{nH(X)}} = e^{n(H(X,Y) - H(X))} = e^{nH(Y|X)}
\end{split}
\end{equation}
\subexercise Compute the probability of $x^n \in T_{P_{X}}$ and $y^n \in T_W(x^n)$ with exponential approximation.
We already know:
\begin{equation} \label{eq4}
\begin{split}
Pr(x^n \in T_{P_X}) \doteq e^{-nH(P_X)}\\
Pr(y^n \in T_{P_Y}) \doteq e^{-nH(P_Y)}\\
Q^n(T_P) \doteq e^{-nD(P||Q)}
\end{split}
\end{equation}
So we have:
\begin{equation} \label{eq5}
\begin{split}
& Pr(x^n \in T_{P_{X}} \And y^n \in T_W(x^n)) \\
& = Pr(x^n \in T_{P_X}) * Pr(y^n \in T_W(x^n)) \\
& = Pr(x^n \in T_{P_X}) * Pr((x^n, y^n) \in T_{P_{XY}} | x^n \in T_{P_X})\\
& = Pr(x^n \in T_{P_X}) * Pr(x^n \in T_{P_X} \And y^n \in T_{P_Y} | x^n \in T_{P_X})\\
& = Pr(x^n \in T_{P_X}) * Pr(x^n \in T_{P_X} | x^n \in T_{P_X}) * Pr(y^n \in T_{P_Y} | x^n \in T_{P_X})\\
& = Pr(x^n \in T_{P_X}) * 1 * P^n_Y(T_{P_X})\\
& \doteq e^{-nH(P_X)} * e^{-nD(P_X||P_Y)} = e^{-n(H(P_X) + D(P_X||P_Y)}
\end{split}
\end{equation}

\end{document}
