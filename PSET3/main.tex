\documentclass[
  % all of the below options are optional and can be left out
  % course name (default: 2IL50 Data Structures)
  course = {{EE623 Information Theory}},
  % quartile (default: 3)
  quartile = {{Fall 2020}},
  % assignment number/name (default: 1)
  assignment = 3,
  % student name (default: Some One)
  name = {{Mohammad Mahdi Rahimi}},
  % student number, NOT S-number (default: 0123456)
  studentnumber = {{20208244}},
  % student email (default: s.one@student.tue.nl)
  email = {{mahi@kaist.ac.kr}},
  % first exercise number (default: 1)
  firstexercise = 1
]{aga-homework}

\usepackage{amssymb,latexsym,amsmath,amsthm}
\usepackage{amsfonts,rawfonts}
\usepackage{thmtools}
\usepackage{systeme}
\usepackage{mathtools,cancel}
\usepackage{tikz}
\usepackage{pgfplots} 
\usepackage[pdf]{graphviz}
\pgfplotsset{width=10cm,compat=1.9} 
 \usepgfplotslibrary{external}

\tikzexternalize 

\begin{document}

\exercise

\subexercise $P_1 \ge P_2 \ge P_3 \ge P_4$ and $P_1 = P_3 + P_4$
\\\\
First choice of Huffman is $P_3 - P_4$. After that we have new $P_1 = P_{34} \ge P_2$ so second choice can be either $(A)\ P_1 - P_2$ or $(B)\ P_{34} - P_2$ different decision leads to following graphs.
\\
(A): All codeword with same length of two\\
(B): a codeword of length 1, one of length 2, and two of length 3.

\subexercise Find $P_{max}$ and $P_1 = P_3 + P_4$
\\\\
\begin{equation} \label{eq3}
\begin{split}
P_1 + P_2 + P_3 + P_4  & = 1\\
\Rightarrow  2P_1 + P_2 & = 1\\
\Rightarrow  P_2 & = 1 - 2P_1
\end{split}
\end{equation}

\begin{equation} \label{eq3}
\begin{split}
& P_1 = P_3 + P_4, P_2 \ge P_3 \ge P_4\\
\Rightarrow & P_1 \le 2P_2\\
(1) \Rightarrow & P_1 \le 2 - 4P_1\\
\Rightarrow & P_1 \le 0.4 \Rightarrow P_{max} = 0.4
\end{split}
\end{equation}

\subexercise Find $P_{min}$ and $P_1 = P_3 + P_4$
\\\\
\begin{equation} \label{eq3}
\begin{split}
& P_2 = 1 - 2P_1, P_1 \ge P_2\\
\Rightarrow & P_1 \ge 1 - 2P_1\\
\Rightarrow & 3P_1 \ge 1\\
\Rightarrow & P_1 \ge {1 \over 3} \Rightarrow P_{min} = {1 \over 3} \simeq 0.333
\end{split}
\end{equation}

\subexercise $P_{1} > P_{max}$, show length of Huffman codeword is 1 
\\\\
First choice of Huffman is $P_3 - P_4$. After that we have new $P_1 \ge P_{34} and P_2$ so second choice is always $\ P_{34} - P_2$ and it leads to a codeword with length 1 for M_1.

\subexercise $P_{1} < P_{min}$, show length of Huffman codeword is 2 
\\\\
First choice of Huffman is $P_3 - P_4$. After that we have new $P_{34} \ge P_1 \ge P_2$ so second choice is always $\ P_1 - P_2$ and it leads to a codeword with length 2 for M_1.

\exercise
\subexercise
\\\\

\begin{equation}
    \begin{split}
        & P^n(X^n \in A^n_\epsilon \cap B^n_\epsilon) \ge 1 - \epsilon\\
        \Rightarrow & P^n(X^n \notin A^n_\epsilon \cap B^n_\epsilon) \le \epsilon\\
        \Rightarrow & P^n(X^n \in A^{n\prime}_\epsilon \cup B^{n\prime}_\epsilon) \le \epsilon\\
        \text{Union Bound} \Rightarrow & P^n(X^n \in A^{n\prime}_\epsilon \cup B^{n\prime}_\epsilon) \le P^n(X^n \in A^{n\prime}_\epsilon) + P^n(X^n \in  B^{n\prime}_\epsilon) \le \epsilon
    \end{split}
\end{equation}

\begin{equation}
    \begin{split}
        & A^{n\prime}_\epsilon = \{x^n \in X^n : |-{1 \over n}\sum_i{\log{P(x_i)}} - H(P)| > \delta\} \\
        AEP \Rightarrow & P(A^{n\prime}_\epsilon) = P(|-{1 \over n}\sum_i{\log{P(x_i)}} - H(P)| > \delta) < \epsilon\\
        \Rightarrow & P^n(X^n \in A^{n\prime}_\epsilon) < \epsilon
    \end{split}
\end{equation}

\begin{equation}
    \begin{split}
        & B^{n\prime}_\epsilon = \{x^n \in X^n : |{1 \over n}\sum_i{x_i} - \mu| > \delta\} \\
        WLLN \Rightarrow & P(B^{n\prime}_\epsilon) = P(|{1 \over n}\sum_i{x_i} - \mu| > \delta) < \epsilon\\
        \Rightarrow & P^n(X^n \in B^{n\prime}_\epsilon) < \epsilon
    \end{split}
\end{equation}

\begin{equation}
    \begin{split}
        (4), (5), (6) \Rightarrow & P^n(X^n \in A^{n\prime}_\epsilon) + P^n(X^n \in  B^{n\prime}_\epsilon) \le \epsilon\\
        \Rightarrow & P^n(X^n \in A^{n\prime}_\epsilon \cup B^{n\prime}_\epsilon) \le \epsilon\\
        \Rightarrow & P^n(X^n \in A^n_\epsilon \cap B^n_\epsilon) \ge 1 - \epsilon\\
        &\text{for any } \epsilon > 0 \text{ there is a n sufficiently large to satisfy}
    \end{split}
\end{equation}

\subexercise
\\\\

From the definition of $A^n_\epsilon$, we have:\\
\begin{equation}
    \begin{split}
        2^{-n(H(P)) + \epsilon} \le P^n(x^n) \le 2^{-n(H(P))-\epsilon}\\
    \end{split}
\end{equation}\\
Therefore,\\
\begin{equation}
    \begin{split}
        |A^n_\epsilon \cap B^n_\epsilon|2^{-n(H(P) + \epsilon} \le |A^n_\epsilon \cap B^n_\epsilon|P^n(x^n) \le |A^n_\epsilon \cap B^n_\epsilon|2^{-n(H(P))-\epsilon}\\
    \end{split}
\end{equation}\\
Since $P^n(|A^n_\epsilon \cap B^n_\epsilon|) \le 1$ then:
\begin{equation}
    \begin{split}
        &|A^n_\epsilon \cap B^n_\epsilon|2^{-n(H(P) + \epsilon} \le 1\\
        \Rightarrow & |A^n_\epsilon \cap B^n_\epsilon| \le 2^{n(H(p) + \epsilon)}
    \end{split}
\end{equation}\\

\subexercise
\\\\
By using (7) and (9) for large enough $n$ we have: $P^n(A^n_\epsilon \cap B^n_\epsilon) \ge 1 - \epsilon$ then:
\begin{equation}
    \begin{split}
        & 1 - \epsilon \le |A^n_\epsilon \cap B^n_\epsilon|2^{-n(H(P))-\epsilon}\\
        \Rightarrow & |A^n_\epsilon \cap B^n_\epsilon| \ge (1 - \epsilon)2^{n(H(p) - \epsilon)}
    \end{split}
\end{equation}\\


\end{document}
