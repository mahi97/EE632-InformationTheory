\documentclass[
  % all of the below options are optional and can be left out
  % course name (default: 2IL50 Data Structures)
  course = {{EE623 Information Theory}},
  % quartile (default: 3)
  quartile = {{4}},
  % assignment number/name (default: 1)
  assignment = 4,
  % student name (default: Some One)
  name = {{Mohammad Mahdi Rahimi}},
  % student number, NOT S-number (default: 0123456)
  studentnumber = {{20208244}},
  % student email (default: s.one@student.tue.nl)
  email = {{mahi@kaist.ac.kr}},
  % first exercise number (default: 1)
  firstexercise = 1
]{aga-homework}

\usepackage{amssymb,latexsym,amsmath,amsthm}
\usepackage{amsfonts,rawfonts}
\usepackage{thmtools}
\usepackage{systeme}
\usepackage{mathtools,cancel}
\usepackage{tikz}
\usepackage{pgfplots} 
\usepackage[pdf]{graphviz}
\pgfplotsset{width=10cm,compat=1.9} 
 \usepgfplotslibrary{external}

\tikzexternalize 

\begin{document}

\exercise
\subexercise provide a way to construct a source code at rate $R = H(X|Y)$.\\


According to $H(X,Y) = H(Y) + H(X|Y)$, by having a side-information, Y, (Y is coded with $R_y \ge H(Y)$) we can reduce the entropy of X from $H(X)$ to $H(X|Y)$ and it leads to reduce number of bits required for coding X to $R = \lceil H(X|Y) \rceil$, in theory.\\

In practice, we define $2^{nR_y}$ bags w.r.t. coding of Y, then we can reduce the problem of coding a sequence of symbols in X with distribution of $P_x$ to a coding with distribution inside the specified bag of Y which is $P(X|Y)$ and corresponding entropy are $H(X|Y)$ so coding with $R = H(X|Y)$ can have small error with sufficient large n.

\\
\exercise
\subexercise Show (2) and (3)
\begin{equation} \label{eq3}
\begin{split}
P^n(T_P) & = (\text{\# of seq. in }T_P) * (\text{Probability to see a seq. of } T_P \text{ over distribution of P})\\
(\textbf{Multinomial distribution})& = \frac{n!}{x_1!, x_2!, ..., x_m!}P^{x_1}_1P^{x_2}_2...P^{x_m}_m. \text{ in which} (x_i = nP(a_i))\\
& = \frac{n!}{\prod_{a \in X} (nP(a))!}\prod_{a \in X} P(a)^{nP(a)}
\end{split}
\end{equation}\\
Likewise,\\
\begin{equation} \label{eq3}
\begin{split}
P^n(T_Q) & = (\text{\# of seq. in }T_Q) * (\text{Probability to see a seq. of } T_Q \text{ over distribution of P})\\
(\textbf{Multinomial distribution})& = \frac{n!}{x_1!, x_2!, ..., x_m!}P^{x_1}_1P^{x_2}_2...P^{x_m}_m. \text{ in which} (x_i = nQ(a_i))\\
& = \frac{n!}{\prod_{a \in X} (nQ(a))!}\prod_{a \in X} P(a)^{nQ(a)}
\end{split}
\end{equation}
\subexercise Show (4)\\
\begin{equation} \label{eq3}
\begin{split}
\frac{P^n(T_Q)}{P^n(T_P)} & = \frac{\frac{n!}{\prod_{a \in X} (nQ(a))!}\prod_{a \in X} P(a)^{nQ(a)}} {\frac{n!}{\prod_{a \in X} (nP(a))!}\prod_{a \in X} P(a)^{nP(a)}}\\
& =  \frac{\prod_{a \in X} (nP(a))!} {\prod_{a \in X} (nQ(a))!} \prod_{a \in X} P(a)^{n(Q(a) - P(a))}\\
& = \prod_{a \in X} \frac{ (nP(a))!} {(nQ(a))!} P(a)^{n(Q(a) - P(a))}
\end{split}
\end{equation}

\subexercise Use (5) to Prove (1)\\
\begin{equation} \label{eq3}
\begin{split}
\frac{n!}{m!} & \le n^{n - m}\\
\text{with }n \rightarrow nP(a), m \rightarrow nQ(a)  \Rightarrow \frac{(nP(a))!}{(nQ(a))!} & \le (nP(a))^{n(P(a) - Q(a))}\\
\Rightarrow \frac{(nQ(a))!}{(nP(a))!} * (P(a))^{n(Q(a) - P(a))} & \le (nP(a))^{n(P(a) - Q(a))} * (P(a))^{n(Q(a) - P(a))}\\
\Rightarrow \frac{(nQ(a))!}{(nP(a))!} * (P(a))^{n(Q(a) - P(a))} & \le n^{n(P(a) - Q(a))}\\
\Rightarrow \prod_{a \in X}\frac{(nQ(a))!}{(nP(a))!} * (P(a))^{nQ(a) - P(a))} & \le  \prod_{a \in X}n^{n(P(a) - Q(a))} = n^{\sum_{a \in X} n(P(a) - Q(a))}\\
\Rightarrow \frac{P^n(T_Q)}{P^n(T_P)} & \le n^{n(\sum P(a) - \sum Q(a))} = n^{n(1 - 1))} = n^0 = 1 \\
\Rightarrow P^n(T_Q) & \le P^n(T_P)
\end{split}
\end{equation}


\end{document}
