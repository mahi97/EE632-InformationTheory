\documentclass[
  % all of the below options are optional and can be left out
  % course name (default: 2IL50 Data Structures)
  course = {{EE623 Information Theory}},
  % quartile (default: 3)
  quartile = {{4}},
  % assignment number/name (default: 1)
  assignment = 6,
  % student name (default: Some One)
  name = {{Mohammad Mahdi Rahimi}},
  % student number, NOT S-number (default: 0123456)
  studentnumber = {{20208244}},
  % student email (default: s.one@student.tue.nl)
  email = {{mahi@kaist.ac.kr}},
  % first exercise number (default: 1)
  firstexercise = 1
]{aga-homework}

\usepackage{amssymb,latexsym,amsmath,amsthm}
\usepackage{amsfonts,rawfonts}
\usepackage{thmtools}
\usepackage{systeme}
\usepackage{mathtools,cancel}
\usepackage{tikz}
\usepackage{pgfplots} 
\usepackage[pdf]{graphviz}
\pgfplotsset{width=10cm,compat=1.9} 
 \usepgfplotslibrary{external}

\tikzexternalize 

\begin{document}

\exercise Finding $P_x$ to maximize channel capacity
\\\\
For finding $P_x$ which maximize $I_{P_x}(P_X, W)$ we use Lagrange Multipliers, and assuming $P_x$ is to form of $[p,q,q]$. 

\begin{equation} \label{eq3}
\begin{split}
\mathcal{L}(P_x, \lambda) = \mathcal{L}(p, q, \lambda) & = I(P_x, W) - \lambda(p + 2q - 1) \\
I(P_x, W) & = H(Y) - H(Y|X)
\end{split}
\end{equation}\\

\begin{equation} \label{eq3}
\begin{split}
H(Y) & =  p\log{p} + 2q\log{q}\\
H(Y|X) & =  Pr(X=1)\cancelto{0}{H(Y|X=1)}+Pr(X=2)\cancelto{H_{Bern}(\epsilon)}{H(Y|X=2)}+Pr(X=3)\cancelto{H_{Bern}(\epsilon)}{H(Y|X=3)}\\
& = 2qH_{Bern}(\epsilon)\\
\Rightarrow I(P_x, W) & = p\log{p} + 2q\log{q} - 2qH_{Bern}(\epsilon)\\
\Rightarrow \mathcal{L}(p, q, \lambda) & = p\log{p} + 2q\log{q} - 2qH_{Bern}(\epsilon) - \lambda(p + 2q - 1)
\end{split}
\end{equation}

\begin{equation} \label{eq3}
\begin{split}
\begin{cases}
\frac{\partial\mathcal{L}}{\partial p} & = \log{p} + 1 - \lambda\\
\frac{\partial\mathcal{L}}{\partial q}  & = 2\log{q} + 2 - 2H_{Bern}(\epsilon) - 2\lambda\\
\frac{\partial\mathcal{L}}{\partial \lambda} & = p + 2q - 1\\
\end{cases}\\
\end{split}
\end{equation}
\begin{equation} \label{eq3}
\begin{split}
& \Rightarrow \begin{cases}
\log{p} + 1 - \lambda = 0\\
2\log{q} + 2 - 2H_{Bern}(\epsilon) - 2\lambda = 0\\
p + 2q - 1 = 0 \Rightarrow p = 1 - 2q\\
\end{cases}\\
& \Rightarrow \begin{cases}
\log{1 -2q} + 1 - \lambda = 0 \Rightarrow 2\lambda - 2\log{1 -2q} - 2 = 0\\
2\log{q} + 2 - 2H_{Bern}(\epsilon) - 2\lambda = 0\\
\end{cases}\\
&\Rightarrow 2\log{q} + 2 - 2H_{Bern}(\epsilon) - 2\log{1 -2q} - 2 = 0\\
&\Rightarrow 2\log{q} - 2H_{Bern}(\epsilon) - 2\log{1 -2q} = 0\\
&\Rightarrow \log{q} - H_{Bern}(\epsilon) - \log{1 -2q} = 0\\
&\Rightarrow \log{\frac{q}{1 -2q}} - H_{Bern}(\epsilon) = 0\\
&\Rightarrow \log{\frac{q}{1 -2q}} = H_{Bern}(\epsilon)\\
&\Rightarrow \frac{q}{1 -2q} = e^{H_{Bern}(\epsilon)}\\
&\Rightarrow q = \frac{e^{H_{Bern}(\epsilon)}}{1 + 2e^{H_{Bern}(\epsilon)}} \Rightarrow 2q = \frac{2e^{H_{Bern}(\epsilon)}}{1 + 2e^{H_{Bern}(\epsilon)}} \Rightarrow p = \frac{1}{1 + 2e^{H_{Bern}(\epsilon)}}\\
\end{split}
\end{equation}

\\
\exercise
\\\\
We want to examine satisfaction of following inequalities
\begin{equation} \label{eq3}
\begin{split}
&\left|- \frac{1}{n} \sum_{i=1}^{n} \log{P_{XY}(x_i, y_i)} - H(P_{XY})\right| \le \epsilon,\\
&\left| - \frac{1}{n} \sum_{i=1}^{n} \log{P_X(x_i)} - H(P_X)\right| \le \epsilon,\\
&\left| - \frac{1}{n} \sum_{i=1}^{n} \log{P_Y(y_i)} - H(P_Y)\right| \le \epsilon,\\
\end{split}
\end{equation}
and we have,
\begin{equation} \label{eq3}
\begin{split}
& P_{XY}(0,0) = \frac{4}{8},P_{XY}(1,1) = \frac{2}{8},P_{XY}(0,1) = P_{XY}(1,0) = \frac{1}{8}. \\
& \hat{P}_{x^ny^n}(0,0) = \frac{6}{16},\hat{P}_{x^ny^n}(1,1) = \frac{8}{16},\hat{P}_{x^ny^n}(0,1) = \hat{P}_{x^ny^n}(1,0) = \frac{1}{16}. 
\end{split}
\end{equation}
then we calculate marginal distribution as:
\begin{equation} \label{eq3}
\begin{split}
& P_{X}(0) = P_{Y}(0) = \frac{5}{8},P_{X}(1) = P_{Y}(1) = \frac{3}{8}. \\
& \hat{P}_{X}(0) = \hat{P}_{Y}(0) = \frac{7}{16},\hat{P}_{X}(1) = \hat{P}_{Y}(1) = \frac{9}{16}. \\
\end{split}
\end{equation}
There for inequalities of (5) can be check as follows with $n \rightarrow \infty$:
\begin{equation} \label{eq3}
\begin{split}
& -\frac{1}{n} \sum_{i=1}^{n} \log{P_{XY}(x_i, y_i)} - H(P_{XY}) = 0 \\
\Rightarrow & \begin{cases}
-\frac{1}{n} \sum_{i=1}^{n} \log{P_{XY}(x_i, y_i)} = \frac{6}{16}\log\frac{4}{8} + \frac{8}{16}\log\frac{2}{8} + \frac{1}{16}\log\frac{1}{8} + \frac{1}{16}\log\frac{1}{8} = \frac{14}{8}\\
H(P_{XY}) = \frac{4}{8}\log\frac{4}{8} + \frac{2}{8}\log\frac{2}{8} + \frac{1}{8}\log\frac{1}{8} + \frac{1}{8}\log\frac{1}{8} = \frac{14}{8}
\end{cases}\\
& \Rightarrow \frac{14}{8} - \frac{14}{8} = 0.\ \textbf{First inequality satisfied}
\end{split}
\end{equation}
\begin{equation} \label{eq3}
\begin{split}
& -\frac{1}{n} \sum_{i=1}^{n} \log{P_{X}(x_i)} - H(P_{X}) = 0 \\
\Rightarrow & \begin{cases}
-\frac{1}{n} \sum_{i=1}^{n} \log{P_{X}(x_i)} = \frac{7}{16}\log\frac{5}{8} + \frac{9}{16}\log\frac{3}{8}\\
H(P_{X}) = \frac{5}{8}\log\frac{5}{8} + \frac{3}{8}\log\frac{3}{8}
\end{cases}\\
& \Rightarrow \left|-\frac{7}{16}\log\frac{5}{8} - \frac{9}{16}\log\frac{3}{8} + \frac{5}{8}\log\frac{5}{8} + \frac{3}{8}\log\frac{3}{8}\right| = \left|\frac{3}{16}\log\frac{5}{8} - \frac{3}{16}\log\frac{3}{8} \right| > 0.\\
&\textbf{Second inequality not satisfied, Likewise Third inequality}
\end{split}
\end{equation}

So it shows that there exist some distribution over $XY$ that satisfy the first inequality but not the others, in other word satisfaction of first inequality cannot grant the satisfaction of second and third inequalities. 

\end{document}
