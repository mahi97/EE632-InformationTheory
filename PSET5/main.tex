\documentclass[
  % all of the below options are optional and can be left out
  % course name (default: 2IL50 Data Structures)
  course = {{EE623 Information Theory}},
  % quartile (default: 3)
  quartile = {{Fall 2020}},
  % assignment number/name (default: 1)
  assignment = 5,
  % student name (default: Some One)
  name = {{Mohammad Mahdi Rahimi}},
  % student number, NOT S-number (default: 0123456)
  studentnumber = {{20208244}},
  % student email (default: s.one@student.tue.nl)
  email = {{mahi@kaist.ac.kr}},
  % first exercise number (default: 1)
  firstexercise = 1
]{aga-homework}

\usepackage{amssymb,latexsym,amsmath,amsthm}
\usepackage{amsfonts,rawfonts}
\usepackage{thmtools}
\usepackage{systeme}
\usepackage{mathtools,cancel}
\usepackage{tikz}
\usepackage{pgfplots} 
\usepackage[pdf]{graphviz}
\pgfplotsset{width=10cm,compat=1.9} 
 \usepgfplotslibrary{external}

\tikzexternalize 

\begin{document}

\exercise
\subexercise
Prove Strong typical set is a subset of Weak typical set

By definition of Strong typical set we have:
\begin{equation} \label{eq1}
\begin{split}
& | P(x) - Q(x) | \le \delta \\
\Rightarrow & Q(x) - \delta \le P(x) \le Q(x) + \delta \\\\
\Rightarrow & -Q(x)log(Q(x)) - \delta log(Q(x)) \le -P(x)log(Q(x)) \le -Q(x)log(Q(x)) + \delta log(Q(x))\\\\
\Rightarrow & \sum{-Q(x)log(Q(x))} - \sum{\delta log(Q(x))} \le \sum{-P(x)log(Q(x))} \le \sum{-Q(x)log(Q(x))} + \sum{\delta log(Q(x))}\\\\
\Rightarrow & H(Q) - \delta \sum{log(Q(x))} \le \sum{-P(x)log(Q(x))} \le H(Q) + \delta \sum{log(Q(x))}\\\\
\Rightarrow & H(Q) - \epsilon(\delta) \le \sum{-P(x)log(Q(x))} \le H(Q) + \epsilon(\delta)
\end{split}
\end{equation}
On the other hand we have,
\begin{equation} \label{eq2}
\begin{split}
& Q(X) = \prod{Q(x_i}^{nP(x_i)}\\\\
\Rightarrow & log(Q(X)) = n\sum{P(x_i)logQ(x_i)}\\\\
\Rightarrow & P(x_i)\sum{logQ(x_i)} = \frac{1}{n}log(Q(X))
\end{split}
\end{equation}\\
By applying (2) in (1) we get:
\begin{equation} \label{eq3}
\begin{split}
H(Q) - \epsilon(\delta) \le -\frac{1}{n}log(Q(X)) \le H(Q) + \epsilon(\delta)\\
\end{split}
\end{equation}\\
According to definition of Weak Typical sets and (3), we can conclude than any Strong Typical set is an Weak Typical set.

\newpage
\exercise

with t satisfying:
\begin{equation} \label{eq4}
\begin{split}
\mathbb{E}_{P_t} [F(X)] & = \sum_{x \in X} P_t(x)F(x) = \eta \\
P_t(x) & = \frac{1}{g(t)}Q(x)e^{tF(x)} \\
g(t) & = \sum{Q(x)e^{tF(x)}}
\end{split}
\end{equation}\\
We show $s = t$ will maximize following term:
\begin{equation} \label{eq5}
\begin{split}
s\eta - \log{g(s)}
\end{split}
\end{equation}
We also have:
\begin{equation} \label{eq6}
\begin{split}
\frac{dg(t)}{dt} = 
g^{\prime}(t) = (\sum{Q(x)e^{tF(x)}})^{\prime} = \sum{Q(x)F(x)e^{tF(x)}}
\end{split}
\end{equation}\\
For this we derivative (5) and equal it to zero:
\begin{equation} \label{eq7}
\begin{split}
& \eta - \frac{g^{\prime}(s)}{g(s)} = 0 \\
\Rightarrow & \eta = \frac{g^{\prime}(s)}{g(s)} \\
(4) \Rightarrow & \sum_{x \in X} P_t(x)F(x) = \frac{g^{\prime}(s)}{g(s)} \\
(4) \Rightarrow & \sum_{x \in X} \frac{1}{g(t)}Q(x)e^{tF(x)}F(x) = \frac{g^{\prime}(s)}{g(s)} \\
\Rightarrow & \frac{1}{g(t)}\sum_{x \in X} Q(x)e^{tF(x)}F(x) = \frac{g^{\prime}(s)}{g(s)} \\
(6) \Rightarrow & \frac{g^{\prime}(t)}{g(t)} = \frac{g^{\prime}(s)}{g(s)}\\
\Rightarrow & s = t
\end{split}
\end{equation}\\
So, \textbf{s} equals to \textbf{t} will maximize (5), the exponent in the Chernoff bound is equal to the exponent that been proved
\end{document}
